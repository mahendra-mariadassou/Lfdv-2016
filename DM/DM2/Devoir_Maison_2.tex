\documentclass[a4paper,12pt]{article}\usepackage[]{graphicx}\usepackage[]{color}
%% maxwidth is the original width if it is less than linewidth
%% otherwise use linewidth (to make sure the graphics do not exceed the margin)
\makeatletter
\def\maxwidth{ %
  \ifdim\Gin@nat@width>\linewidth
    \linewidth
  \else
    \Gin@nat@width
  \fi
}
\makeatother

\definecolor{fgcolor}{rgb}{0.345, 0.345, 0.345}
\newcommand{\hlnum}[1]{\textcolor[rgb]{0.686,0.059,0.569}{#1}}%
\newcommand{\hlstr}[1]{\textcolor[rgb]{0.192,0.494,0.8}{#1}}%
\newcommand{\hlcom}[1]{\textcolor[rgb]{0.678,0.584,0.686}{\textit{#1}}}%
\newcommand{\hlopt}[1]{\textcolor[rgb]{0,0,0}{#1}}%
\newcommand{\hlstd}[1]{\textcolor[rgb]{0.345,0.345,0.345}{#1}}%
\newcommand{\hlkwa}[1]{\textcolor[rgb]{0.161,0.373,0.58}{\textbf{#1}}}%
\newcommand{\hlkwb}[1]{\textcolor[rgb]{0.69,0.353,0.396}{#1}}%
\newcommand{\hlkwc}[1]{\textcolor[rgb]{0.333,0.667,0.333}{#1}}%
\newcommand{\hlkwd}[1]{\textcolor[rgb]{0.737,0.353,0.396}{\textbf{#1}}}%
\let\hlipl\hlkwb

\usepackage{framed}
\makeatletter
\newenvironment{kframe}{%
 \def\at@end@of@kframe{}%
 \ifinner\ifhmode%
  \def\at@end@of@kframe{\end{minipage}}%
  \begin{minipage}{\columnwidth}%
 \fi\fi%
 \def\FrameCommand##1{\hskip\@totalleftmargin \hskip-\fboxsep
 \colorbox{shadecolor}{##1}\hskip-\fboxsep
     % There is no \\@totalrightmargin, so:
     \hskip-\linewidth \hskip-\@totalleftmargin \hskip\columnwidth}%
 \MakeFramed {\advance\hsize-\width
   \@totalleftmargin\z@ \linewidth\hsize
   \@setminipage}}%
 {\par\unskip\endMakeFramed%
 \at@end@of@kframe}
\makeatother

\definecolor{shadecolor}{rgb}{.97, .97, .97}
\definecolor{messagecolor}{rgb}{0, 0, 0}
\definecolor{warningcolor}{rgb}{1, 0, 1}
\definecolor{errorcolor}{rgb}{1, 0, 0}
\newenvironment{knitrout}{}{} % an empty environment to be redefined in TeX

\usepackage{alltt}
%%%%%%%%%%%%%%%%%%%%%%%%%%%%%%%%%%%%%%%%%%%%%%%%%%%%%%%%%%%%%%%%%%%%
%%% ZONE ROUGE : intervention très fortement deconseillee
\usepackage[utf8]{inputenc} % accent dans la source
\usepackage[T1]{fontenc}
\usepackage{times}
\usepackage[french]{babel}
\usepackage{amsmath}
\usepackage{amsfonts}
\usepackage{amssymb}
\usepackage{fancybox}
\usepackage{pstricks}
\usepackage{pst-plot}
% \usepackage{pst-char}
\usepackage{pst-text}
% \usepackage{psfig}
\usepackage{ifthen}
\def\R{\mathbb{R}}
\def\C{\mathbb{C}}
\def\Z{\mathbb{Z}}
\def\Q{\mathbb{Q}}
\def\N{\mathbb{N}}
\def\F{\mathbb{F}}
\def\P{\mathbb{P}}
\def\A{\mathbb{A}}

%%%% Théorème, Proposition et tout le reste%%%%
\newcommand{\proofbegin}{\paragraph{Proof.}}
\newcommand{\proofend}{$\blacksquare$\bigskip}
\newtheorem{theorem}{Théorème}
\newtheorem{proposition}[theorem]{Proposition}
\newtheorem{definition}[theorem]{Définition}
\newtheorem{lemma}[theorem]{Lemme}
\newtheorem{corollary}[theorem]{Corollaire}


%%%%%%%%%%%%%%%%%%%%%%%%%%%%%%%%%%%%%%%%%%%%%%%%%%%%%%%%%%%%%%%%%%%%%
%%% ZONE ORANGE : intervention deconseillee
\parindent=0pt
\textwidth 17.0cm
\textheight26.0cm
\hoffset-2.0cm
\voffset-2.0cm
%%%%%%%%%%%%%%%%%%%%%%%%%%%%%%%%%%%%%%%%%%%%%%%%%%%%%%%%%%%%%%%%%%%%%
%%% ZONE VERTE : Intervention obligatoire
%%% ADAPTER SUIVANT LA NATURE DE L'EPREUVE
\def\Exam{Devoir Maison \og Complexes \fg{}}
\def\Date{2 décembre 2016}
\def\Classe{TS3}

%%%%%%%%%%%%%%%%%%%%%%%%%%%%%%%%%%%%%%%%%%%%%%%%%%%%%%%%%%%%%%%%%%%%%
%%% ZONE VERTE : 
%%% ALIGNER LES EQUATIONS A GAUCHE
\makeatletter
\newenvironment*{fleqn}{
    \@fleqntrue
    \setlength\@mathmargin{0pt}%
    \ignorespaces
}{%
    \ignorespacesafterend
}
\makeatother

%%%%%%%%%%%%%%%%%%%%%%%%%%%%%%%%%%%%%%%%%%%%%%%%%%%%%%%%%%%%%%%%%%%%%
%%% knitr



%%%%%%%%%%%%%%%%%%%%%%%%%%%%%%%%%%%%%%%%%%%%%%%%%%%%%%%%%%%%%%%%%%%%%
\IfFileExists{upquote.sty}{\usepackage{upquote}}{}
\begin{document}
\newcounter{nexo}
\setcounter{nexo}{1}
\newcommand{\Exo}{\medskip
  {\bf Exercice \arabic{nexo} : }
  \addtocounter{nexo}{1}}
\newcommand{\Pb}{{\bf Problème \arabic{nexo} : } 
\addtocounter{nexo}{1} \bigskip}
%%%%%%%%%%%%%%%%%%%%%%%%%%%%%%%%%%%%%%%%%%%%%%%%%%%%%%%%%%%%%%%%%%%%%
%%% ZONE BLEUE : Intervention parfois utile mais a faire prudemment
{\bf  \hfill \Date \quad ~}
%
\vskip 1cm
%
%%% FAIRE UN CHOIX (3 choix possibles)
%%% 1)
%\centerline{\psframebox[fillstyle=solid,fillcolor=lightgray]
%\textbf{\LARGE \black \Exam}}
%%% 2)
%\centerline{\psframebox{\bf \LARGE  \Exam}}
%%% 3)
\centerline{\bf \LARGE \Exam}
%
\vskip 1.5cm
%


%%%%%%%%%%%%%%%%%%%%%%
%%% CORPS DU SUJET %%%
%%%%%%%%%%%%%%%%%%%%%%

Ce devoir maison est \textbf{faculatif}. Il a vocation à \emph{rattraper} les notes un peu basses du quizz sur les complexes selon des modalités à l'étude (choix entre note du quizz et moyenne du quizz et du DM, note du quizz augmenté de X points en fonction des résultats du DM, etc). 

Les exercices 1 à 3 ont une cohérence thématique (autour de l'astuce du demi-angle) et ont vocation à vous faire travailler l'exponentielle complexe et manipulation de puissances. Il est \textbf{vivement} recommandé de les faire dans l'ordre (le 3 est le plus difficile du point de vue \emph{calculatoire}, il ne sera pas noté et est uniquement là pour les gens qui veulent s'entraîner). Les exercices 4 à 7 sont indépendants et de difficultés comparables. 

\Exo \textbf{Astuce du demi-angle}

\begin{enumerate}
\item Soit $\theta \in \mathbb{R}$ un nombre complexe. En factorisant par $e^{i\theta/2}$, donner un argument et le module de 
$$z_1 = 1 + e^{i\theta} = e^{i\theta/2} (e^{-i\theta/2} + e^{i\theta/2})$$
On rappelle que le module d'un nombre complexe est $\textbf{positif}$.
\item Donner de même un argument et un module de 
$$z_2 = 1 - e^{i\theta}$$
\end{enumerate}

\Exo \textbf{Complexes et Binôme}

Soit $\theta \in \mathbb{R}$ un nombre et $n$ un entier naturel. On veut calculer les quantités $R_n$ et $I_n$ définies par 
\begin{equation*}
I_n  = \sum_{k=0}^n {n \choose k} \cos(k\theta) \text{ et } R_n  = \sum_{k=0}^n {n \choose k} \sin(k\theta)
\end{equation*}

\begin{enumerate}
\item Montrer que $\displaystyle S_n = R_n + i I_n = (1+e^{i\theta})^n$
\item En utlisant l'astuce du demi-angle, simplifier l'expression précédente. On cherchera à faire apparaître le produit d'un nombre réel (pas forcément le module) et d'un nombre complexe de module 1. 
\item En déduire $R_n$ et $I_n$
\end{enumerate}

\Exo \textbf{Complexes et Courants (facultatif, non noté)}

Soit $\theta \in \mathbb{R}$ un nombre et $n$ un entier naturel. On veut calculer les quantités $R_n$ et $I_n$ définies par 
\begin{equation*}
I_n  = \sum_{k=0}^n \cos(k\theta) \text{ et } R_n  = \sum_{k=0}^n \sin(k\theta)
\end{equation*}

\begin{enumerate}
\item Montrer que $\displaystyle S_n = R_n + i I_n = \frac{1 - e^{i(n+1)\theta}}{1 - e^{i\theta}}$
\item En utlisant l'astuce du demi-angle, simplifier l'expression précédente. On cherchera à faire apparaître le produit d'un nombre réel (pas forcément le module) et d'un nombre complexe de module 1. 
\item En déduire 
$$ R_n = \frac{\cos\left(\frac{n\theta}{2}\right)\sin\left(\frac{(n+1)\theta}{2}\right)}{\sin\left(\frac{\theta}{2}\right)} \text{ et } I_n = \frac{\sin\left(\frac{n\theta}{2}\right)\sin\left(\frac{(n+1)\theta}{2}\right)}{\sin\left(\frac{\theta}{2}\right)} $$
\end{enumerate}

NB: Ces quantités interviennent quand on superpose plusieurs courants qui ont des fréquences croissantes (par exemple $10$ Hz, $20$ Hz, $30$ Hz, etc) et qu'on veut connaître le courant résultant.

\Exo \textbf{Complexes et Linéarisation}

Linéariser $\sin(x)\cos^3(x)$

\Exo \textbf{Complexes et Plan Local d'Urbanisme}

En consultant le Plan Local d'Urbanisme, on a trouvé une maison en forme de parallélépipède rectangle (boîte à chaussure) de volume $504$m$^3$, de surface au sol $168$m$^2$ et de périmétre au sol $52$m. Quelles sont ses dimensions?

\Exo \textbf{Complexes et Racine Carrée}

Trouver une racine carré de $Z = 3+4i$. Il \textbf{faut} passer par la méthode algébrique ($Z$ n'a pas de forme exponentielle simple).

\Exo \textbf{Complexes et Crustacés}

Soit $\displaystyle A = \exp\left( \frac{1 + 2i}{100} \right)$. On considère la suite complexe $(z_n)$ définie par $z_0 = 0.1$ et $z_{n+1} = A z_n$ pour tout $n \geq 0$.
On cherche à exprimer $x_n = Re(z_n)$ et $y_n = Im(z_n)$ en fonction de $n$. 
\begin{enumerate}
\item Exprimer $z_n$ en fonction de $n$, $z_0$ et $A$ uniquement. 
\item En déduire $x_n$ et $y$ (sous forme simple!). 
\end{enumerate}


Bonus: si vous avez un grapheur, vous pouvez essayer d'afficher la trajectoire de la suite dans le plan complexe (de $n=0$ à $n=300$ par exemple). 

\end{document}
